\section{Điều chỉnh cơ chế sinh số ngẫu nhiên để tăng độ chính xác}

Trong bài toán Monte Carlo, chất lượng và cách sử dụng số ngẫu nhiên ảnh hưởng trực tiếp đến:
\begin{itemize}
    \item Độ \textbf{ổn định} của kết quả giữa các lần chạy.
    \item Độ \textbf{chính xác} của giá trị xấp xỉ $\pi$ khi số điểm \verb|nPoints| tăng lên.
\end{itemize}



Sử dụng seed cố định và phụ thuộc vào chỉ số toàn cục

Thay vì khởi tạo seed từ thời gian hệ thống (\verb|time(NULL)|), nhóm sử dụng một \emph{seed gốc} cố định \verb|base_seed| và xây dựng seed cho từng điểm dựa trên chỉ số toàn cục $i$:
\begin{verbatim}
unsigned int seed = base_seed + (unsigned int)i;
\end{verbatim}

Nhờ đó:
\begin{itemize}
    \item Với mỗi điểm sẽ có 1 seed duy nhất, giúp phân tán tốt hơn các điểm ngẫu nhiên.
    \item Với cùng giá trị \verb|nPoints|, mỗi chỉ số $i$ luôn tạo ra cùng một cặp $(x, y)$, không phụ thuộc vào số luồng hay cách phân chia công việc.
    \item Tập các điểm ngẫu nhiên được sinh ra trong phiên bản đơn luồng và đa luồng là như nhau (chỉ khác cách phân chia giữa các luồng).
    \item Chạy lại chương trình nhiều lần với cùng cấu hình luôn cho cùng một kết quả xấp xỉ $\pi$, giúp việc so sánh, đo \textit{speedup} và phân tích trở nên đáng tin cậy hơn.
\end{itemize}


Nhờ việc ``làm mềm'' cơ chế ngẫu nhiên theo hướng \emph{deterministic nhưng vẫn đủ phân tán}, nhóm vừa đảm bảo được tính ngẫu nhiên cho thuật toán Monte Carlo, vừa đảm bảo được tính lặp lại và độ chính xác cao cho kết quả thực nghiệm.
